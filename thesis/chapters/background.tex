\chapter{Literature review}

This chapter should give the reader an overview of the components required for a global illumination model.
For the sake of the scope, one will briefly describe the different topics; therefore, if concepts or terms are unclear, one has to revise those accordingly.
As a side note, the order of sections does not correspond to the linearity of the concepts.

\section{Light Integrals}

\subsection{Relevance of Physics}

\subsection{Rendering Equation}

\subsection{Lighting Equation}

\subsection{Path Equation}

\section{Monte Carlo Integration}

One encounters complex mathematical operators in computer science, leading to numerous issues.
Integrals do count to these operators bringing many challenges with their variety in form and dimensionality.
Hence, approximations with numerical methods are the only solution to evaluate such terms due to the finite resources computers have.
The numerical method requires that it should be able to handle high dimensional and discontinuous functions as well as allow for modifications to make it more robust.
\cite{pharr_physically_2017}
The term robust will be defined later after the introduction of  Monte-Carlo integration.

Path-Tracing uses the Monte-Carlo integration because it satisfies the previously mentioned requirements. 
As with all Monte-Carlo techniques, it does use randomness for repeated random sampling to find an accurate solution on average.
Therefore, one gives a quick overview of probability to understand how the integration method works.

\subsection{Probability Theory}

A \textbf{random variable} is a function that maps a random process to a number assumed to be real-valued, and one denotes it with a capitalised letter.
The \textbf{cumulative distribution function} $F(x)$ gives the probability for a random variable $X$ that is less than or equal to a value $x$.

\begin{align*}
F(X)=Pr\{X\le x\}
\end{align*}

Using calculus, one can differentiate $F(x)$ yielding the \textbf{probability density function} $f(x)$, only if such solution exists.
The density function describes the likelihood for a random variable to be in an interval $[a,b]$.

\begin{align*}
f(x)=\frac{dF(x)}{dx}
\end{align*}

The expected value is a weighted average, which is evaluable for a random variable.
This requires the cumulative distribution $F(x)$ and corresponding density function $f(x)$.

\begin{align*}
E[X]=\int_{\Omega}F(x)\,f(x)\,dx
\end{align*}

The final metric is the variance which gives the mean squared deviation.

\begin{align*}
V[X]=E\left[(x-E[x])^2\right]
\end{align*}

\subsection{Estimator}

\subsection{Optimisation}


\cite{veach_robust_nodate}